\section{Conception du programme} % (fold)
\label{sec:conception}

La parallélisation du programme de lancer de rayons s'est faite en plusieurs étape. Le programme se compose d'une scène qui est elle même séparée en tuiles. Chaque tuile correspond à un carré de pixels. Le calcul de chaque tuile est indépendant du calcul des autres tuiles. Grace à cette propriété, il est possible de paralléliser le programme uniquement en s'occupant de la répartition du calcul des tuiles. La bibliothèque MPI sera utilisée pour ce travail.\\

Dans un premier temps, nous avons utilisé le théorème des reste chinois pour répartir les tuiles sur les différents processus MPI. Cela permet d'avoir une répartition pseudo aléatoire des tuiles. Pour $P$ processus MPI, le programme est constitué d'un processus principal qui rassemblera le travail des $P-1$ processus secondaires, qui effectuerons le calcul des tuiles. Dans un second temps, nous avons séparé la charge de travail au sein des processus secondaires. Nous avons pour cela utilisé les threads POSIX. Chaque processus crée un nombre de threads $t$ défini à l'avance. Le travail des processus est alors séparé entre les threads grace à une file de travail. Chaque thread récupère une tuile à calculer jusqu'à ce que la file soit épuisée.\\

Le théorème des restes chinois réparti de façon équitable le travail en fonction du nombre de tuiles à calculer. Cependant, le temps de calcul des tuiles varie selon la complexité des rayons. Pour équilibrer la charge de calcul entre les processus, nous avons créé un anneau de communication entre les processus secondaires. Lorsqu'un processus n'a plus de travail, il envoi un message sur l'anneau. Si un processus possédant encore du travail à réaliser reçoit une demande de travail, il envoie directement le travail au processus demandant le travail S'il n'a pas de travail à transmettre, il envoi la requête de travail au processus suivant dans l'anneau de communication. Lorsqu'un processus reçoit son propre message de demande de travail, cela signifie que toutes les tuiles sont calculées ou en cours de calcul. Il envoi alors un message pour signaler la fin du programme et se termine. Un processus qui reçoit le message de terminaison sait qu'il n'y a plus de travail à effectuer et envoi à son tour le message de terminaison et se termine une fois le calcul de ses tuiles accompli.

\begin{figure}[H]
\centering
\includegraphics[width=0.6\textwidth]{automate.jpg}
\caption{Automate d'un processus secondaire pour les communications réseau}
\label{fig:diff}
\end{figure}

Afin de tester le bon fonctionnement de la transmission des taches, nous avons implémenté un système de taches factices permettant de créer des taches de longueur variable au lieu du calcul des tuiles. Le but est de créer un nombre différent de taches sur chaque processus et de vérifier que la charge de calcul est la même sur chaque processus.

% /section end
